% cybersecurity is important
% enumeration of cybersecurity issues among different softwares, and typical solutions
% mobile securities -- access control, mention our approach: security type system based verification, we have a prototype for our type system, and a taint-based checking implementation
% native binary security -- level is much lower, mention our approach: fuzz testing
% summarization

Software security has been growing in importance due to the increasing reliance on computer systems, the internet, and the popularity of smartphones as well as other devices that constitute the Internet of things. Due to the complexity, software security is also one of the major challenges of contemporary world. Modern softwares usually have multiple vulnerabilities that indicate a security weakness in design, implementation, operation, or internet control. There are various categories of vulnerabilities, including the denial-of-service attacks which result from unexpected software crashes, information leakage caused by memory access violations or design defects, privilege escalation caused by misconfigured access control mechanism. There are typically two categories of approaches to securing a system by reducing its surface of vulnerability, namely the verification and the testing. The former models the system in an abstracted description and proves its preservations of certain properties, this is usually applies on the detection of higher level vulnerability detection such as access control. The latter is more widely used in software development procedures to reveal multi-level vulnerabilities, which uses multiple test cases to guarantee that the underlying software works properly under certain scenarios. In this thesis, we will describe our efforts in both approaches in securing various softwares.

We apply the verification approach on the Android access control. 