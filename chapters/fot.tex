% !TeX root =../../main.tex

\chapter{The FOT Fuzzing Framework} \label{ch:fot}


\section{Introduction and Motivation}


In spite of the popularity and effectiveness of applying grey-box fuzzing techniques in detecting vulnerabilities(c.f. Sec~\ref{sec:intro-gbf}), there lacks a fuzzing framework to easily reuse, integrate and compare different fuzzing extensions and~experiment with new ideas.
Take {\AFL} as an example, it is implemented~all in one file with around 8K LOC, which contains more than 100 global variables.
Hence, the implementation of a single feature often involves modifications in multiple places.
In short, {\AFL} is compact but also highly coupled because {\AFL} is designed to \textit{require essentially no configuration}~\cite{afl}.
In fact, most of the existing fuzzers are designed for easy deployment and usage, but not easy extension.
Therefore, it is desirable to have a fuzzing framework that allows easy \emph{configuration} and \emph{extension} for new features.


To this end, we propose our fuzzing framework, namely \emph{Fuzzing Orchestration Toolkit}  ({\FOT}). {\FOT} is designed to hold three properties.

\begin{enumerate}[(1)]


\item  \textbf{Versatility.}
{\FOT} provides a fuzzing ecosystem, including a set of static and dynamic analyses used to aid the fuzzing process.


\item \textbf{Configurability.}
{\FOT} provides a set of configurable options.	
Users can easily tweak the parameters of the fuzzer to improve the fuzzing effectiveness with their experience.

\item \textbf{Extensibility.}
{\FOT} is designed to be of high coherence and low coupling. Specially, the implementation mainly consists of two parts: the library containing general fuzzing utilities and miscellaneous tools on top of it. Therefore, apart from the default fuzzer provided by {\FOT}, developers can write their own fuzzers with modest effort based on the library.
\end{enumerate}


\section{Architecture Design}\label{sec:details}

In this section, we describe the design of {\FOT} framework.
More technical details of {\FOT} are available at ~\cite{fot-webpage}.


\begin{figure}[t]
	\centering
	\includegraphics[width=0.6\textwidth]{res/fot/FOT_overview}
	\caption{Overview of the {\FOT} Fuzzing Framework}
	\label{fig:fot_workflow}
\end{figure}

Figure~\ref{fig:fot_workflow} depicts the overview of {\FOT}.
It consists of three parts, namely the \emph{preprocessor}, the \emph{fuzzer}, and the \emph{complementary toolchain}.
Components of the framework are represented with \emph{blue} rectangles while the inputs and outputs are in \emph{grey}. All these components inside \FOT are \textit{configurable} and \textit{extensible}.


\subsection{Preprocessor}
This part contains various tools for collecting static information and instrumentation with the PUT.


\subsubsection{Static Analyzer}\label{sec:static_analysis}
This includes various tools to extract semantic understandings from the PUT.
For example, we have tools to generate the control flow graph, call graph or statically collected vulnerability information and convert them into suitable representations that can later be instrumented into the PUT and utilized during the fuzzing process.
This part is \textit{configurable} to generate different levels of static information. It is \textit{extensible} as developers are allowed to add new types of static analysis as long as the generated result follows the specified format.


\subsubsection{Instrumentor}
The \emph{binary rewriter} and the \emph{compiler instrumentor} instrument additional static information generated by the static analyzer into the PUT so that the fuzzer can collect feedback from the latter during execution.
{\FOT} supports Dyninst~\cite{dyninst} based instrumentation when only binary is provided, and LLVM based instrumentation when the source code is available.
This part is \textit{configurable} as the users can choose to either instrument at source code level or at binary level.
It is \textit{extensible} since developers can use other tools such as Intel Pin~\cite{pin} for instrumentation as long as the instrumented code can embed the static information and follow the regulations to provide feedback for the fuzzer.

\subsection{Fuzzer}
This part explains {\FOT}'s the main fuzzing process. 
It is essentially a loop that continuously selects seeds from the queue, applies mutations to the selected seeds, executes the PUT against mutated inputs, and collects feedback for the next iteration.

\subsubsection{Conductor}
As {\FOT} is designed to support multi-threaded parallel fuzzing, it contains the \emph{conductor} for fuzzing, managing the workload of each worker thread.
Particularly, it can listen to a special directory to actively import seed inputs from external sources such as symbolic executors like KLEE~\cite{klee} or mutation generators like Radamsa~\cite{radamsa}.
This part is \textit{configurable} as the users can choose different strategies for the overall management.
It is \textit{extensible} as it can interoperate with other seed generation tools.


\subsubsection{Seed Scorer}
The seed scorer is in charge of selecting a seed from the queue for mutation (seed prioritization) and determining how many new inputs should be generated based on the selected seed (power scheduling).
This part is \textit{configurable} as the users can select from several built-in scoring strategies to evaluate seeds.
It is \textit{extensible} as the users can implement their own strategies with the interfaces provided in {\FOT}.


\subsubsection{Mutation Manager}
The mutation manager is in charge of incorporating different mutators.
It can mutate the seeds in a pure random manner or according to predefined specifications.
This part is \textit{configurable} as {\FOT} provides various mutators for the users to choose from.
It is \textit{extensible} as the developers can implement their own mutators with the provided library.

\subsubsection{Executor}
The executor drives the execution of the PUT.
This part is \textit{configurable} as the default executor in {\FOT} allows users to choose whether or not to use forkserver~\cite{afl} during fuzzing.
It is \textit{extensible} as the developers can extend the executor for different scenarios.
For example, they may add a secondary executor to execute another PUT to perform differential testing.

\subsubsection{Feedback Collector}
The feedback collector collects the feedback emitted by the PUT.
The exact feedback often corresponds to the instrumented information.
This part is \textit{configurable} as the users are allowed to select from the default feedback options provided by {\FOT}.
For now, the feedback can be at basic-block level (like {\AFL}) or function level.
It is \textit{extensible} as the users can specify their customized types of feedback for collection.

\subsection{Complementary Toolchain}
{\FOT} additionally contains various tools helping to make the framework \textit{versatile}.
For instance, we implemented a web-based frontend user interface to monitor the fuzzing results.
It provides fruitful information to make the fuzzing process more transparent.
We also implemented a crash analyzer to analyze the detected crashes and generate reports automatically.
This reduces the manual efforts of crash triaging.
In addition, many more extensions are being added to complement the fuzzer.


\begin{figure}[t]
	\centering
	\includegraphics[width=0.6\columnwidth]{res/fot/mt_workflow}
	\caption{Conductor with Multiple Fuzzing Instances}
	\label{fig:mt_workflow}
\end{figure}


\subsection{Trace Update and Synchronization}\label{sec:trace_sync}
One of {\FOT}'s key features is the builtin coordination between different fuzzing instances (Fig.~\ref{fig:mt_workflow}). One important issue is the trace synchronization between fuzzing instances. For instrumentation, we made a slight change to the conventional instrumentation runtime to make the target binaries able to distinguish different shared memory arenas and the file descriptors used for ``forkserver'' allocated/specified by different fuzzing workers. Each fuzzing worker allocates the shared memory and the instrumented binary writes to specific multiple 8-byte areas when the corresponding ``execution edges'' have been reached. By auditing the byte fingerprints, the fuzzer knows about the edges and their approximate hit counts within this run. This information sits between between ``branch coverage'' and ``path coverage''. By comparing the shared memory fingerprint with the local trace information (checking whether the active shared memory byte has been marked ``traced'' locally), the fuzzer gets the knowledge whether current running seed increases the coverage. Updating of the local trace is majorly a ``bitwise and'' where each byte of the local trace is initialized with all ones (i.e., 255).

The local trace is synchronized with the global trace state. There stands a tradeoff: if we use directly the global trace state, the synchronization will be too frequent and eventually decreases performance with the increase of more fuzzing workers; if the fuzzers are only aware of the local trace, it is no better than {\AFL}'s na\"ive approach that runs all instances separately. We thus choose to only apply the synchronization during the mutation of each test case in the queue, when customizable conditions are triggered (usually the conditions are about the executions and time since last synchronization).

The actual synchronization of the trace information still applies ``bitwise and'' on normal running traces
%~\footnote{In practice, we also synchronize the timeout traces and crash traces to avoid generating too many redundant ``abnormal'' test cases.}
from local trace information to global state, and an instant copy in the other direction. This is far more efficient than {\AFL}'s synchronization by 1) importing seeds from other directories and 2) running all the test cases indistinguishably. Note that {\FOT}'s synchronization does not lose precisions compared to {\AFL}'s, where both ``bitwise and'' operations erase the exact hit count information.


  \subsection{Mutation Strategy Adaption}\label{sec:mutation_ops}
 The selection of mutation operators during fuzzing on one test input is determined by two factors:
 \begin{enumerate}
 	\item The whitelist mutation operators used for the targeted binaries. Some mutation operators are only effective on certain programs, but is almost a waste of time for the other programs (for example, bitflip operations are quite expensive and rarely useful for text-based parser programs running against large files). This can be specified by the experienced {\FOT} users.
 	\item The one-time mutation operators for \emph{this fuzz}. This is automatically determined by the fuzzer according to statistics generated from the previous mutations and runs. It is calculated in an adaptive way and may finally help to determine the ``convergence'' of the test cases.
 \end{enumerate}

On the other hand, being super general, {\AFL} has limited configurations for what mutation operators can be used and frequently runs blindly on the mutations that do not fit well~\cite{junjie:2017sp:skyfire,mopt-fuzz}.


\subsection{Refinement on Variable Behaviors}\label{sec:entry_var_behavior}

Some programs have variable behaviors for the same input test, due to randomness, multi-threading, etc. {\AFL} handles this issue by running all the newly found interesting test cases multiple times (known as \emph{calibration}); whenever it finds that the shared memory information (the active bytes and their hit count) differs from the first run, it will give more chances to the running test entry, and then keeps track of the variable behavior rate. The problem is that it does not utilize the information further since variable behaviors may cause certain runs to exit normally at some time, however crash at other time, which is more serious. We separately track those test case and give even more chances for these seeds. Alternatively, we provide an intrinsic strategy to prioritize these cases and let those test cases to be more likely to run next time. On the other hand, {\AFL}'s tracing information for the variable behavior cases are imprecise since it only traces the last calibration shared memory; we refine this by (selective) ``bitwise or'' operations to the shared memory for subsequent procedures on the current seed.


 \subsection{Trimming on duplicated cases}
Due to the existence of the potential lag of the trace synchronization, there still exists test input that runs with the same running trace. In other cases, the test cases might not be in its ``simplest'' form: by removing some bytes, the input test case can still results in the same running trace. {\FOT}'s approach in handling this is to trim the calibrated test cases immediately before being parceled as the message and sent to the message queue buffer managed by the conductor. And the conductor maintains a checksum set of all the generated seed files. Therefore when the newly generated seed has the same checksum as one of the existing ones, this test case will be discarded.

Additionally, we provide an external minimizer program to prune the all the serialized test cases (which can be normal runs, timeout runs, or crashes); it is more aggressive than the embedded procedure in the fuzzer and aims to provide a minimized version of all the interesting test cases.
 

%\section{Implementation and Extensions}\label{sec:app}

%We have implemented the {\FOT} framework and developed several extensions to {\FOT}.

\subsection{Implementation}\label{sec:fot-impl}

The {\FOT} project started from June, 2017 and has been actively developed by two researchers. It is implemented with 16000 lines of Rust for core fuzzing modules, together with 4100 lines of C/C++ for the preprocessor, 4800 lines of Java for structure-aware mutation, and 2400 lines of Python for complementary toolchain.



%\subsection{Static Vulnerability Analysis Integration}\label{subsec:sva}
%
%
%
%Grey-box fuzzers are typically aware of quantitative changes of code coverage and use such feedback for keeping the \textit{interesting} seeds.
%However, the performance of collecting code coverage feedback quantitatively is often not ideal, and grey-box fuzzers also need to evaluate the code coverage qualitatively~\cite{Bohme:2016:CGF}.
%One of the approaches to bring qualitative awareness about the covered code is to combine fuzzing with static vulnerability analysis, as mentioned in \S~\ref{sec:static_analysis}.
%
%\begin{figure}[t]
%	\centering
%	\includegraphics[width=0.75\columnwidth]{res/fot/moo_result.pdf}
%	\caption{Average Number of Unique Crashes Found in 24 Hours on \textit{mjs} and \textit{intel-xed}.}
%	\label{fig:moo_result}
%\end{figure}
%
%
%Existing fuzzing works seldom use static analysis information to facilitate seed prioritization and power scheduling since existing fuzzing frameworks have little support for them.
%In contrast, integration with vulnerability static analysis is trivial in \FOT: we used \emph{static analyzer} to calculate vulnerability metrics (e.g., calls~to unsafe functions and cyclomatic complexity) and customized \emph{seed scorer} to take them into account during fuzzing. This extension added about 330 lines of C++ and 190 lines of Rust code.
%
%The workflow is as follows.
%First, the static analyzer calculates vulnerability score for each function. Then we instrument the PUT to provide function level coverage information.
%After detecting a seed that brings new coverage, the feedback collector will collect its function level coverage and map it with the static analysis result to get the function level vulnerability scores.
%The seed scorer will then accumulate the function-level scores to form the execution trace level vulnerability scores for the exercised seeds.
%Finally, the power schedule determined by the seed scorer prioritizes and allocates more powers to the seeds with higher vulnerability scores.
%
%
%
%Figure~\ref{fig:moo_result} shows the average number of unique crashes detected on mjs and intel-xed of different fuzzers over 10 runs.
%We can see that with the help of static vulnerability analysis, {\FOT} can detect more unique crashes in a limited time budget.

  

\section{Comparisons to Other Fuzzing Frameworks and Extensions}


In this section, we first compare FOT with other fuzzing frameworks, and then discuss its relationship to current fuzzing extensions.



Table~\ref{tbl:cmp_fuzz} compares {\FOT} with existing fuzzing frameworks with respect to 10 major features. As we can see, the existing fuzzing frameworks AFL, libFuzzer and honggfuzz lack features in different aspects, while {\FOT} integrates all of them. {\FOT} stands out in that it provides various configurations for advanced users; it is also highly modularized to be easily extended with other fuzzing techniques. Further more, {\FOT} also partially supports structure-aware mutations (by specifying semantic grammars) and interoperability with other seed generation tools such as symbolic executors (by monitoring and scheduling newly incoming seed input directory).

Most current fuzzing techniques can be easily integrated into {\FOT} thanks to its highly-modularized design. In fact, these techniques can be applied with some extensions to the different components in Figure~\ref{fig:fot_workflow} and can be used together with the configuration interface. 

\begin{enumerate}[1)]
	\item AFLFast~\cite{Bohme:2016:CGF} can be implemented by applying a Markov Chain model based seed \emph{power scheduling} in the fuzzer. 
	\item AFLGo~\cite{Bohme:2017:DGF} can be implemented by a combination of \emph{static analyzer}, \emph{instrumentation} and \emph{power scheduling}.
	\item CollAFL~\cite{CollAFL} can be implemented by using a collision-resistant algorithm to increase the uniqueness of the path trace labeling during \emph{instrumentation}.
	\item Skyfire~\cite{junjie:2017sp:skyfire}, Radamsa~\cite{radamsa}, Csmith~\cite{csmith} can be used in the preprocessor to generate seeds for the \emph{external seeds} and a structure-aware mutator assigned by \emph{mutation manager}.
	\item Symbolic executors such as KLEE~\cite{klee} can be integrated in the Driller's~\cite{driller} style with the help of \emph{conductor}.
\end{enumerate}

\begin{table}[t]
\centering
	\small
	\caption{Comparisons between Different Fuzzing Frameworks (\Circle: not supported, \LEFTcircle: partially supported, \CIRCLE: fully supported)}
	\label{tbl:cmp_fuzz}
	\begin{tabular}{|l|c|c|c|c|}
		\hline
		\diagbox{\textbf{Features}}{\textbf{Framework}} & \textbf{AFL} & \textbf{libFuzzer} & \textbf{honggfuzz} & \textbf{FOT} \\ \hline\hline
		Binary-Fuzzing Support & \CIRCLE & \Circle & \CIRCLE & \CIRCLE \\ \hline
		Multi-threading Mode & \Circle & \CIRCLE  & \CIRCLE  & \CIRCLE  \\ \hline
		In-memory Fuzzing &\CIRCLE  & \CIRCLE &\CIRCLE  & \CIRCLE \\ \hline
		Advanced Configuration & \Circle  & \LEFTcircle  & \Circle  & \CIRCLE  \\ \hline
		Modularized Functionality & \Circle & \LEFTcircle & \Circle & \CIRCLE \\ \hline
		Structure-aware Mutation & \Circle  &\Circle & \Circle  & \LEFTcircle \\ \hline
		Interoperability & \Circle & \Circle & \Circle & \LEFTcircle \\
		\hline
		Toolchain Support &  \CIRCLE & \Circle  & \Circle  & \CIRCLE \\ \hline
		Precise Crash Analysis & \Circle  & \Circle  & \CIRCLE  & \CIRCLE  \\ \hline
		Runtime Visualization & \LEFTcircle & \Circle & \Circle & \CIRCLE \\ \hline
	\end{tabular}
\end{table}


\subsection{New Vulnerabilities}

Till now, {\FOT} has been used to fuzz more than 100 widely used open source projects and it has detected more than 200 vulnerabilities in these projects, among these 51 CVE IDs have been assigned. The detailed information is available at Appendix~\ref{app:bugs}. Notably, \FOT has detected multiple vulnerabilities with \emph{high} or \emph{critical} severity. For example, CVE-2019-9169~\cite{CVE-2019-9169} stresses a vulnerability in the GNU C Library (aka glibc or libc6) through 2.29, where the function \func{proceed\_next\_node} inside POSIX regular expression (regex) component has a heap-based buffer over-read via an attempted case-insensitive regex match. Thanks to the interoperability with the regex generators, FOT is able to generate more meaningful seeds that cover more corner cases of the regex components. According to CVSS v3.0 Severity and Metrics~\cite{cvss3}, it is scored 9.8 out of 10.0 (critical severity); according to CVSS v2.0~\cite{cvss2}, it has a score of 7.5 (high severity)~\footnote{As a comparison, the notorious HeartBleed (CVE-2014-0160) is scored 5.0 medium severity according to CVSS v2.0; there is no CVSS v3.0 for it.}. Many of other vulnerabilities discovered by \FOT also have high severity, e.g, CVE-2018-15822, CVE-2018-14394, CVE-2018-14395 in FFmpeg, CVE-2018-19837, CVE-2018-19838, CVE-2018-19839, CVE-2018-20821, CVE-2018-20822 in libsass, CVE-2018-14560, CVE-2018-14561, CVE-2019-11470, CVE-2019-11472, CVE-2019-11473 in ImageMagick, and CVE-2019-11474 in GraphicsMagick. We envision that with more fuzzing techniques, \FOT will have the capabilities to detect more vulnerabilities that are hard to be discovered by existing fuzzers.
